\hypertarget{CLASSBD_8f}{}\section{C\+L\+A\+S\+S\+B\+D.\+f File Reference}
\label{CLASSBD_8f}\index{C\+L\+A\+S\+S\+B\+D.\+f@{C\+L\+A\+S\+S\+B\+D.\+f}}


Purpose\+: Assign values to parameters in C\+L\+A\+S\+S common blocks. C\+L\+A\+S\+S incorporates several kinds of parameters in its common blocks. Some are defined specifically for use in the C\+L\+A\+S\+S code; some are also shared with the atmospheric model (if running in coupled mode).  




\subsection{Detailed Description}
Purpose\+: Assign values to parameters in C\+L\+A\+S\+S common blocks. C\+L\+A\+S\+S incorporates several kinds of parameters in its common blocks. Some are defined specifically for use in the C\+L\+A\+S\+S code; some are also shared with the atmospheric model (if running in coupled mode). 

In routine C\+L\+A\+S\+S\+B\+D, values are primarily assigned to the parameters that are specific to the C\+L\+A\+S\+S code and are passed through it via common blocks C\+L\+A\+S\+S1 through C\+L\+A\+S\+S8. The table below lists the scalar parameters, their definitions, and their designated values with units. \[ \begin{tabular} { | l | l | c | c | } \hline Name & Definition & Value & Units \\ \hline VKC & Von Karman Constant & 0.40 & - \\ \hline CT & Drag Coefficient for water & $1.15 10^{-3}$ & - \\ \hline VMIN & Minimum wind speed & 0.1 & $ m s^{-1} $ \\ \hline TCW & Thermal conductivity of water & 0.57 & $W m^{-1} K^{-1} $ \\ \hline TCICE & Thermal conductivity of ice & 2.24 & $W m^{-1} K^{-1} $ \\ \hline TCSAND & Thermal conductivity of sand particles & 2.5 & $W m^{-1} K^{-1} $ \\ \hline TCCLAY & Thermal conductivity of fine mineral particles & 2.5 & $W m^{-1} K^{-1} $ \\ \hline TCOM & Thermal conductivity of organic matter & 0.25 & $W m^{-1} K^{-1} $ \\ \hline TCDRYS & Thermal conductivity of dry mineral soil & 0.275 & $W m^{-1} K^{-1} $ \\ \hline RHOSOL & Density of soil mineral matter & $2.65 10^3 $ & $kg m^{-3} $ \\ \hline RHOOM & Density of soil organic matter & $1.30 10^3 $ & $kg m^{-3} $ \\ \hline HCPW & Volumetric heat capacity of water & $4.187 10^6 $ & $J m^{-3} K^{-1} $ \\ \hline HCPICE & Volumetric heat capacity of ice & $1.9257 10^6$ & $J m^{-3} K^{-1} $ \\ \hline HCPSOL & Volumetric heat capacity of mineral matter & $2.25 10^6 $ & $J m^{-3} K^{-1} $ \\ \hline HCPOM & Volumetric heat capacity of organic matter & $2.50 10^6 $ & $J m^{-3} K^{-1} $ \\ \hline HCPSND & Volumetric heat capacity of sand particles & $2.13 10^6 $ & $J m^{-3} K^{-1} $ \\ \hline HCPCLY & Volumetric heat capacity of fine mineral particles & $2.38 10^6 $ & $J m^{-3} K^{-1} $ \\ \hline SPHW & Specific heat of water & $4.186 10^3 $ & $J kg^{-1} K^{-1}$ \\ \hline SPHICE & Specific heat of ice & $2.10 10^3 $ & $J kg^{-1} K^{-1}$ \\ \hline SPHVEG & Specific heat of vegetation matter & $2.70 10^3 $ & $J kg^{-1} K^{-1}$ \\ \hline RHOW & Density of water & $1.0 10^3 $ & $kg m^{-3} $ \\ \hline RHOICE & Density of ice & $40.917 10^3$ & $kg m^{-3} $ \\ \hline TCGLAC & Thermal conductivity of ice sheets & 2.24 & $W m^{-1} K^{-1} $ \\ \hline CLHMLT & Latent heat of freezing of water & $0.334 10^6 $ & $J kg^{-1} $ \\ \hline CLHVAP & Latent heat of vaporization of water & $2.501 10^6 $ & $J kg^{-1} $ \\ \hline ZOLNG & Natural log of roughness length of soil & -4.605 & - \\ \hline ZOLNS & Natural log of roughness length of snow & -6.908 & - \\ \hline ZOLNI & Natural log of roughness length of ice & -6.215 & - \\ \hline ZORATG & Ratio of soil roughness length for momentum to roughness length for heat & 3.0 & - \\ \hline ALVSI & Visible albedo of ice & 0.95 & - \\ \hline ALIRI & Near-infrared albedo of ice & 0.73 & - \\ \hline ALVSO & Visible albedo of organic matter & 0.05 & - \\ \hline ALIRO & Near-infrared albedo of organic matter & 0.30 & - \\ \hline ALBRCK & Albedo of rock & 0.27 & - \\ \hline \end{tabular} \].

Values are also assigned to several non-\/scalar parameters, as follows\+:

1) The crop growth descriptor array G\+R\+O\+W\+Y\+R (see the documentation for subroutine A\+P\+R\+E\+P);

2) Three parameters for the four main vegetation categories recognized by C\+L\+A\+S\+S (needleleaf trees, broadleaf trees, crops and grass)\+: Z\+O\+R\+A\+T, the ratio of the roughness length for momentum to the roughness length for heat (currently set to 1); C\+A\+N\+E\+X\+T, an attenuation coefficient used in calculating the sky view factor for vegetation canopies (variable c in the documentation for subroutine C\+A\+N\+A\+L\+B); and X\+L\+E\+A\+F, a leaf dimension factor used in calculating the leaf boundary resistance (variable Cl in the documentation for subroutine A\+P\+R\+E\+P);

3) Six hydraulic parameters associated with the three basic types of organic soils (fibric, hemic and sapric)\+: T\+H\+P\+O\+R\+G, T\+H\+R\+O\+R\+G, T\+H\+M\+O\+R\+G, B\+O\+R\+G, P\+S\+I\+S\+O\+R\+G and G\+R\+K\+S\+O\+R\+G (see the documentation for subroutine C\+L\+A\+S\+S\+B). The table below lists their values, derived from the work of Letts et al. (2000), alongside the symbols used in the C\+L\+A\+S\+S\+B documentation.

\[ \begin{tabular} { | l | l | c | c | c | } \hline Name & Symbol & Fibric peat & Hemic peat & Sapric peat \\ \hline THPORG & $\theta_p $ & 0.93 & 0.88 & 0.83 \\ \hline BORG & b & 2.7 & 6.1 & 12.0 \\ \hline GRKSORG & $K_{sat} $ & $2.8 10^{-4}$ & $2.0 10^{-6}$ & $1.0 10^{-7}$ \\ \hline PSISORG & $psi_{sat} $ & 0.0103 & 0.0102 & 0.0101 \\ \hline THMORG & $theta_{min}$ & 0.04 & 0.15 & 0.22 \\ \hline THRORG & $theta_{ret}$ & 0.275 & 0.62 & 0.705 \\ \hline \end{tabular} \]

Finally, if C\+L\+A\+S\+S is being run offline, values must be assigned to the parameters listed in the first table which would normally be assigned in the G\+C\+M\+:

\[ \begin{tabular} { | l | l | c | c | } \hline GCM Name & Definition & Value & Units \\ \hline DELTIM & Time Step & Varies by run & s \\ \hline CELZRO & Freezing point of water & 273.16 & K \\ \hline GAS & Gas constant & 287.04 & $J kg^{-1} K^{-1}$ \\ \hline GASV & Gas constant for water vapour & 461.50 & $J kg^{-1} K^{-1}$ \\ \hline G & Acceleration due to gravity & 9.80616 & $m s^{-1} $ \\ \hline SIGMA & Stefan-Boltzmann constant & $5.66796 10^{-8}$ & $W m^{-2} K^{-4} $ \\ \hline CPRES & Specific heat of air & $1.00464 10^3 $ & $J kg^{-1} K^{-1}$ \\ \hline CPI & Pi & 3.14159265 & - \\ \hline \end{tabular} \]

The parameters in common blocks P\+H\+Y\+C\+O\+N and C\+L\+A\+S\+S\+D2 that do not have corresponding values assigned in the G\+C\+M are assigned their R\+P\+N values, and the remaining parameters in the G\+C\+M common blocks P\+A\+R\+A\+M\+S, P\+A\+R\+A\+M1, P\+A\+R\+A\+M3 and T\+I\+M\+E\+S are assigned dummy values. 